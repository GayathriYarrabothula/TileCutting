\documentclass{beamer}

\mode<presentation>
{
  \usetheme{default}     
  \usecolortheme{default} 
  \usefonttheme{default} 
  \setbeamertemplate{navigation symbols}{}
  \setbeamertemplate{caption}[numbered]
} 

\usepackage[english]{babel}
\usepackage[utf8]{inputenc}
\usepackage[T1]{fontenc}

\title[Your Short Title]{\textbf{\Huge TILE CUTTING}}
\author{\textbf{BATCH--38}\newline\newline
Y.Gayathri 21B01A05J5 CSE\newline
Y.Sai Varshitha 21B01A05J4 CSE\newline
P.Divya Tejaswini 21B01A05C9 CSE\newline
P.Akhila 21B01A12C9 
IT\newline}

\institute{\Large Shri Vishnu Engineering College for Women}
\date{FEBRUARY 10,2023}

\begin{document}

\begin{frame}
  \titlepage
  
\end{frame}


\section{Python }

\begin{frame}{\Huge Introduction }

\begin{itemize}
\Large
  
  \item To find the maximum possible ways to cut a rectangular shaped tile into parallelograms with the specified area. And finding the area for which the maximum number of possibilities occur in the given range.
\end{itemize}

\end{frame}

\section{Some \LaTeX{} Examples}

\subsection{Tables and Figures}

\begin{frame}{\Huge Approach}

\begin{itemize}
\Large

\item We approached the problem through Brute Force Technique.
\item In order to use python efficiently , we have used built-in methods.
\item To find the area of parallelogram with the given constraints, we have created a list with possible combinations that sum up to give required area.
\end{itemize}



\end{frame}

\subsection{Mathematics}
\begin{frame}{\Huge Learnings}
\begin{itemize}
   

\Large
\item We have learned about sys module to give the input through command line arguments.
\item We got to know the git commands.
\item Through this project we improved our problem solving skills.
\item Teamwork teaches essential communication and social skills.

\end{itemize}


\end{frame}
    


\begin{frame}{\Huge Challenges}
\begin{itemize}
\Large
    \item We initially ran into logical problems while attempting to solve the 
          problem,with the help of our team members we overcome the logical
          errors.
    \item At the very beginning we faced challenges on understanding the problem ,later on with the individual insights we had a complete idea on the problem statement.
\end{itemize}



\end{frame}
\begin{frame}{\Huge STATISTICS}
\begin{itemize}
\Large
    \item The Python code has overall 37 lines.
    \item The code has four user-defined function.\\That  is:\\
    \hspace{0.8cm}1.noOfDivisors()\\
    \hspace{0.8cm}2.createList()\\
    \hspace{0.8cm}3.noOfPossibilities()\\
    \hspace{0.8cm}4.printMaxm()\\
    \item The code has built-in Methods.\\They are:\\
    \hspace{0.8cm}1.append()\\
    \hspace{0.8cm}2.items()\\  
\end{itemize}
    
\end{frame}
\begin{frame}{\Huge Demo/Screenshots}
\begin{figure}[htp]
  \centering
  \includegraphics[width=10cm]{figures/code.png}
   \caption{Code of the project}
   \label{fig:code}
   \end{figure}

    
 \end{frame}
\begin{frame}{\Huge Demo/Screenshots}
\begin{figure}[htp]
  \centering
  \includegraphics[width=10cm]{figures/code1.png}
   \caption{Code of the project}
   \label{fig:code2}
   \end{figure}

    
\end{frame}

\begin{frame}{\Huge Demo/Screenshots}
\begin{figure}[htp]
  \centering
  \includegraphics[width=10cm]{figures/output.png}
   \caption{Sample output}
   \label{fig:output}
   \end{figure}

    
 \end{frame}
\begin{frame}{}
\centering
\textbf{\Huge Thank You}
    
\end{frame}

\end{document}